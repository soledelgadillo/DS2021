% Options for packages loaded elsewhere
\PassOptionsToPackage{unicode}{hyperref}
\PassOptionsToPackage{hyphens}{url}
%
\documentclass[
]{article}
\usepackage{amsmath,amssymb}
\usepackage{lmodern}
\usepackage{ifxetex,ifluatex}
\ifnum 0\ifxetex 1\fi\ifluatex 1\fi=0 % if pdftex
  \usepackage[T1]{fontenc}
  \usepackage[utf8]{inputenc}
  \usepackage{textcomp} % provide euro and other symbols
\else % if luatex or xetex
  \usepackage{unicode-math}
  \defaultfontfeatures{Scale=MatchLowercase}
  \defaultfontfeatures[\rmfamily]{Ligatures=TeX,Scale=1}
\fi
% Use upquote if available, for straight quotes in verbatim environments
\IfFileExists{upquote.sty}{\usepackage{upquote}}{}
\IfFileExists{microtype.sty}{% use microtype if available
  \usepackage[]{microtype}
  \UseMicrotypeSet[protrusion]{basicmath} % disable protrusion for tt fonts
}{}
\makeatletter
\@ifundefined{KOMAClassName}{% if non-KOMA class
  \IfFileExists{parskip.sty}{%
    \usepackage{parskip}
  }{% else
    \setlength{\parindent}{0pt}
    \setlength{\parskip}{6pt plus 2pt minus 1pt}}
}{% if KOMA class
  \KOMAoptions{parskip=half}}
\makeatother
\usepackage{xcolor}
\IfFileExists{xurl.sty}{\usepackage{xurl}}{} % add URL line breaks if available
\IfFileExists{bookmark.sty}{\usepackage{bookmark}}{\usepackage{hyperref}}
\hypersetup{
  pdftitle={ReadMe},
  pdfauthor={Sol Delgadillo},
  hidelinks,
  pdfcreator={LaTeX via pandoc}}
\urlstyle{same} % disable monospaced font for URLs
\usepackage[margin=1in]{geometry}
\usepackage{graphicx}
\makeatletter
\def\maxwidth{\ifdim\Gin@nat@width>\linewidth\linewidth\else\Gin@nat@width\fi}
\def\maxheight{\ifdim\Gin@nat@height>\textheight\textheight\else\Gin@nat@height\fi}
\makeatother
% Scale images if necessary, so that they will not overflow the page
% margins by default, and it is still possible to overwrite the defaults
% using explicit options in \includegraphics[width, height, ...]{}
\setkeys{Gin}{width=\maxwidth,height=\maxheight,keepaspectratio}
% Set default figure placement to htbp
\makeatletter
\def\fps@figure{htbp}
\makeatother
\setlength{\emergencystretch}{3em} % prevent overfull lines
\providecommand{\tightlist}{%
  \setlength{\itemsep}{0pt}\setlength{\parskip}{0pt}}
\setcounter{secnumdepth}{-\maxdimen} % remove section numbering
\ifluatex
  \usepackage{selnolig}  % disable illegal ligatures
\fi

\title{ReadMe}
\author{Sol Delgadillo}
\date{7/23/2021}

\begin{document}
\maketitle

\hypertarget{getting-and-cleaning-data-course-project-human-activity-recognition-using-smartphones-data-set}{%
\subsection{Getting and Cleaning Data Course Project: Human Activity
Recognition Using Smartphones Data
Set}\label{getting-and-cleaning-data-course-project-human-activity-recognition-using-smartphones-data-set}}

The purpose of this project is to demonstrate your ability to collect,
work with, and clean a data set. The goal is to prepare tidy data that
can be used for later analysis.

\hypertarget{data-set}{%
\subsubsection{Data Set}\label{data-set}}

The raw data set can be downloaded from the following website:
{[}\url{http://archive.ics.uci.edu/ml/datasets/Human+Activity+Recognition+Using+Smartphones\#}{]}

Further information on how the data was read and manipulated can be
found in the \href{CodeBook.Rmd}{CodeBook}.

\hypertarget{files}{%
\subsubsection{Files}\label{files}}

\begin{itemize}
\item
  CodeBook.Rmd: contains a description of the data, teh variables and
  any transformations or work performed to clean up the data.
\item
  run\_analysis.R: a script that performs the following:

  \begin{itemize}
  \item
    Merges the training and the test sets to create one data set.
  \item
    Extracts only the measurements on the mean and standard deviation
    for each measurement.
  \item
    Uses descriptive activity names to name the activities in the data
    set.
  \item
    Appropriately labels the data set with descriptive variable names.
  \item
    From the data set in the previous step, creates a second,
    independent tidy data set with the average of each variable for each
    activity and each subject.
  \end{itemize}
\end{itemize}

\end{document}
